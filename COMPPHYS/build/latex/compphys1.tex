%% Generated by Sphinx.
\def\sphinxdocclass{report}
\documentclass[letterpaper,10pt,english]{sphinxmanual}
\ifdefined\pdfpxdimen
   \let\sphinxpxdimen\pdfpxdimen\else\newdimen\sphinxpxdimen
\fi \sphinxpxdimen=.75bp\relax
\ifdefined\pdfimageresolution
    \pdfimageresolution= \numexpr \dimexpr1in\relax/\sphinxpxdimen\relax
\fi
%% let collapsible pdf bookmarks panel have high depth per default
\PassOptionsToPackage{bookmarksdepth=5}{hyperref}

\PassOptionsToPackage{warn}{textcomp}
\usepackage[utf8]{inputenc}
\ifdefined\DeclareUnicodeCharacter
% support both utf8 and utf8x syntaxes
  \ifdefined\DeclareUnicodeCharacterAsOptional
    \def\sphinxDUC#1{\DeclareUnicodeCharacter{"#1}}
  \else
    \let\sphinxDUC\DeclareUnicodeCharacter
  \fi
  \sphinxDUC{00A0}{\nobreakspace}
  \sphinxDUC{2500}{\sphinxunichar{2500}}
  \sphinxDUC{2502}{\sphinxunichar{2502}}
  \sphinxDUC{2514}{\sphinxunichar{2514}}
  \sphinxDUC{251C}{\sphinxunichar{251C}}
  \sphinxDUC{2572}{\textbackslash}
\fi
\usepackage{cmap}
\usepackage[T1]{fontenc}
\usepackage{amsmath,amssymb,amstext}
\usepackage{babel}



\usepackage{tgtermes}
\usepackage{tgheros}
\renewcommand{\ttdefault}{txtt}



\usepackage[Bjarne]{fncychap}
\usepackage{sphinx}

\fvset{fontsize=auto}
\usepackage{geometry}


% Include hyperref last.
\usepackage{hyperref}
% Fix anchor placement for figures with captions.
\usepackage{hypcap}% it must be loaded after hyperref.
% Set up styles of URL: it should be placed after hyperref.
\urlstyle{same}


\usepackage{sphinxmessages}
\setcounter{tocdepth}{1}



\title{COMPPHYS1}
\date{Aug 15, 2022}
\release{}
\author{Yavar\_Azar}
\newcommand{\sphinxlogo}{\vbox{}}
\renewcommand{\releasename}{}
\makeindex
\begin{document}

\ifdefined\shorthandoff
  \ifnum\catcode`\=\string=\active\shorthandoff{=}\fi
  \ifnum\catcode`\"=\active\shorthandoff{"}\fi
\fi

\pagestyle{empty}
\sphinxmaketitle
\pagestyle{plain}
\sphinxtableofcontents
\pagestyle{normal}
\phantomsection\label{\detokenize{index::doc}}


\sphinxAtStartPar
What is computational physics?

\sphinxAtStartPar
“Computational physics is the study and implementation of numerical analysis to solve problems in physics for which a quantitative theory already exists. Historically, computational physics was the first application of modern computers in science, and is now a subset of computational science. It is sometimes regarded as a subdiscipline (or offshoot) of theoretical physics, but others consider it an intermediate branch between theoretical and experimental physics \sphinxhyphen{} an area of study which supplements both theory and experiment”.  \sphinxstyleemphasis{from WIKIPEDIA}


\bigskip\hrule\bigskip


\sphinxstepscope


\chapter{Python Basics}
\label{\detokenize{Python/Python:python-basics}}\label{\detokenize{Python/Python::doc}}

\section{Installation}
\label{\detokenize{Python/Python:installation}}

\section{Python IDEs}
\label{\detokenize{Python/Python:python-ides}}
\sphinxAtStartPar
The integrated development environments (IDEs)


\section{Data Types in Python}
\label{\detokenize{Python/Python:data-types-in-python}}

\subsection{Integers}
\label{\detokenize{Python/Python:integers}}

\subsection{Floats}
\label{\detokenize{Python/Python:floats}}

\subsection{Strings}
\label{\detokenize{Python/Python:strings}}

\subsection{Boolean Type}
\label{\detokenize{Python/Python:boolean-type}}

\subsection{Lists and Tuple}
\label{\detokenize{Python/Python:lists-and-tuple}}

\subsection{Dictionary}
\label{\detokenize{Python/Python:dictionary}}
\sphinxstepscope


\chapter{Numerical}
\label{\detokenize{Numerical/Numerical:numerical}}\label{\detokenize{Numerical/Numerical::doc}}

\section{Arrays}
\label{\detokenize{Numerical/Numerical:arrays}}

\subsection{Getting Into Shape: Array Shapes and Axes}
\label{\detokenize{Numerical/Numerical:getting-into-shape-array-shapes-and-axes}}

\subsection{Data Science Operations: Filter, Order, Aggregate}
\label{\detokenize{Numerical/Numerical:data-science-operations-filter-order-aggregate}}

\section{Numpy}
\label{\detokenize{Numerical/Numerical:numpy}}

\subsection{Functions}
\label{\detokenize{Numerical/Numerical:functions}}

\subsection{Linear algebra}
\label{\detokenize{Numerical/Numerical:linear-algebra}}
\sphinxstepscope


\chapter{Visualization}
\label{\detokenize{Visualization/Visualization:visualization}}\label{\detokenize{Visualization/Visualization::doc}}
\sphinxstepscope


\chapter{Basic Mathematic Operation}
\label{\detokenize{basicmath/basicmath:basic-mathematic-operation}}\label{\detokenize{basicmath/basicmath::doc}}

\section{Discretization}
\label{\detokenize{basicmath/basicmath:discretization}}\begin{equation*}
\begin{split}f_{n}=f\left(x_{n}\right) ; x_{n}=n h(n=0, \pm 1, \pm 2, \ldots),\end{split}
\end{equation*}
\noindent\sphinxincludegraphics[width=400\sphinxpxdimen]{{discr}.jpg}


\section{Differentiation}
\label{\detokenize{basicmath/basicmath:differentiation}}
\sphinxAtStartPar
We begin by using a Taylor series to expand \$f\$ in the neighborhood of \$x=0\$ :
\begin{equation*}
\begin{split}f(x)=f_{0}+x f^{\prime}+\frac{x^{2}}{2 !} f^{\prime \prime}+\frac{x^{3}}{3 !} f^{\prime \prime \prime}+\ldots\end{split}
\end{equation*}
\sphinxAtStartPar
where all derivatives are evaluated at x=0. It is then simple to verify that
\begin{equation*}
\begin{split}\begin{aligned}
&f_{\pm 1} \equiv f(x=\pm h)=f_{0} \pm h f^{\prime}+\frac{h^{2}}{2} f^{\prime \prime} \pm \frac{h^{3}}{6} f^{\prime \prime \prime}+\mathcal{O}\left(h^{4}\right) \\
&f_{\pm 2} \equiv f(x=\pm 2 h)=f_{0} \pm 2 h f^{\prime}+2 h^{2} f^{\prime \prime} \pm \frac{4 h^{3}}{3} f^{\prime \prime \prime}+\mathcal{O}\left(h^{4}\right)
\end{aligned}\end{split}
\end{equation*}\begin{description}
\sphinxlineitem{where \(\mathcal{O}\left(h^{4}\right)\) means terms of order \(h^{4}\) or higher.}\begin{description}
\sphinxlineitem{To estimate the size of such terms, we can assume that  and its derivatives are all of the same order of magnitude,}
\sphinxAtStartPar
as is the case for many functions of physical relevance.

\end{description}

\end{description}
\begin{equation*}
\begin{split}f^{\prime}=\frac{f_{1}-f_{-1}}{2 h}-\frac{h^{2}}{6} f^{\prime \prime \prime}+\mathcal{O}\left(h^{4}\right)\end{split}
\end{equation*}

\bigskip\hrule\bigskip

\begin{equation*}
\begin{split}f^{\prime} \approx \frac{f_{1}-f_{-1}}{2 h}\end{split}
\end{equation*}

\bigskip\hrule\bigskip


\begin{sphinxadmonition}{note}{Note:}
\sphinxAtStartPar
This “3\sphinxhyphen{}point” formula would be exact if  were a second\sphinxhyphen{}degree polynomial in the 3\sphinxhyphen{}point interval \sphinxstyleemphasis{{[}\sphinxhyphen{}h,h{]}},
because the third\sphinxhyphen{} and all higherorder derivatives would then vanish.
\end{sphinxadmonition}

\sphinxAtStartPar
It is more accurate (by one order in  ) than the forward or backward difference formulas:
\begin{quote}
\begin{equation*}
\begin{split}\begin{aligned}
f^{\prime} & \approx \frac{f_{1}-f_{0}}{h}+\mathcal{O}(h) ; \\
f^{\prime} & \approx \frac{f_{0}-f_{-1}}{h}+\mathcal{O}(h) .
\end{aligned}\end{split}
\end{equation*}\end{quote}

\sphinxAtStartPar
extract thiese formula as a practice
\begin{equation*}
\begin{split}f^{\prime} \approx \frac{1}{12 h}\left[f_{-2}-8 f_{-1}+8 f_{1}-f_{2}\right]+\mathcal{O}\left(h^{4}\right) \\
f^{\prime \prime} \approx \frac{f_{1}-2 f_{0}+f_{-1}}{h^{2}} .\end{split}
\end{equation*}

\subsection{Hands\sphinxhyphen{}On}
\label{\detokenize{basicmath/basicmath:hands-on}}

\section{Numerical quadrature}
\label{\detokenize{basicmath/basicmath:numerical-quadrature}}

\section{Finding roots}
\label{\detokenize{basicmath/basicmath:finding-roots}}
\sphinxstepscope


\chapter{Errors}
\label{\detokenize{Errors/Errors:errors}}\label{\detokenize{Errors/Errors::doc}}
\sphinxAtStartPar
\textasciitilde{}To err is human, to forgive divine\textasciitilde{}


\section{Types of Errors}
\label{\detokenize{Errors/Errors:types-of-errors}}

\section{Round\sphinxhyphen{}off Errors}
\label{\detokenize{Errors/Errors:round-off-errors}}

\section{Numerical Recursion}
\label{\detokenize{Errors/Errors:numerical-recursion}}

\section{Error Assessment}
\label{\detokenize{Errors/Errors:error-assessment}}
\sphinxstepscope


\chapter{Integration}
\label{\detokenize{Integration/Integration:integration}}\label{\detokenize{Integration/Integration::doc}}
\sphinxstepscope


\chapter{Derivatives}
\label{\detokenize{Derivatives/Derivatives:derivatives}}\label{\detokenize{Derivatives/Derivatives::doc}}
\sphinxstepscope


\chapter{Matrices}
\label{\detokenize{Matrices/Matrices:matrices}}\label{\detokenize{Matrices/Matrices::doc}}
\sphinxstepscope


\chapter{Data Fitting}
\label{\detokenize{DataFitting/DataFitting:data-fitting}}\label{\detokenize{DataFitting/DataFitting::doc}}
\sphinxstepscope


\chapter{Ordinary differential equation}
\label{\detokenize{ODE/ODE:ordinary-differential-equation}}\label{\detokenize{ODE/ODE::doc}}
\sphinxAtStartPar
A linear differential equation:
\begin{equation*}
\begin{split}a_{0}(x)y+a_{1}(x)y'+a_{2}(x)y''+\cdots +a_{n}(x)y^{(n)}+b(x)=0\end{split}
\end{equation*}
\sphinxAtStartPar
A simple initial value problem
\begin{equation*}
\begin{split}y'=f(x,y),\quad y(x_0)=y_0\end{split}
\end{equation*}
\sphinxAtStartPar
We are interested in computing approximate values of the solution of above Equation in an interval \([x_0,b]\). Thus
\begin{equation*}
\begin{split}x_i=x_0+ih,\quad i=0,1, \dots,n \\
h={b-x_0\over n}\end{split}
\end{equation*}

\section{Numerov Algorithm for Schrödinger ODE}
\label{\detokenize{ODE/ODE:numerov-algorithm-for-schrodinger-ode}}
\sphinxstepscope


\chapter{Projects}
\label{\detokenize{Projects/Projects:projects}}\label{\detokenize{Projects/Projects::doc}}

\chapter{Indices and tables}
\label{\detokenize{index:indices-and-tables}}\begin{itemize}
\item {} 
\sphinxAtStartPar
\DUrole{xref,std,std-ref}{genindex}

\item {} 
\sphinxAtStartPar
\DUrole{xref,std,std-ref}{modindex}

\item {} 
\sphinxAtStartPar
\DUrole{xref,std,std-ref}{search}

\end{itemize}



\renewcommand{\indexname}{Index}
\printindex
\end{document}